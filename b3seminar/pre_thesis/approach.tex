本章では, 本研究で用いたデータと本稿の構成を述べる.

\subsection{使用データ}\label{data}
調音運動を撮像した rtMRI 資料を用いる. 日本語の発話練習を想定し, 日本語を発音している資料を使用する.

このrtMRI資料とは, 前川らが作成した日本語音声 rtMRI 動画データベース\cite{rtMRI}に収録されているものである. MRI の撮影は, 3T MRI システム (Siemens MAGNETOM Prisma fit 3T) が使用され, ATR 脳活動イメージングセンタで行われた. 正中矢状面における調音運動を, 加速係数 3 の FLASH シーケンスを用いて収録されたものである. 解像度は 256 × 256 ピクセル, ピクセルサイズは 1mm × 1mm, スライス厚は 10mm, 時間的再構成率は毎秒 27 フレームである.

正規化処理では個人差を捉えて音素ごとの調音運動パターンの比較を可能とするため, 発話内容が単一音素の資料を用いた. 分析の際には, 単一音素セットの中でも特に /a, u, o, k, t/ の5つの音素を発音した画像と, 発音していない状態の画像の計6種類を用いた. これは, /a, u, o/ で舌の位置と唇の状態の, /k, t/で調音点の, 無音状態で口の開き具合の違いによって, 正規化処理に精度の違いを生じないかを考慮するためである. 男女2名ずつ, 計4名の画像を用いて検討をおこなう.

強調表示処理では音素ごとの運動を捉えるため, 日本語音素の中でも特に運動が大きいカ行とサ行を発音している資料を用いた. この資料のテキストは全て "これがカ〜型" というテンプレートに従って発音されたものである. 以下に示す.
\begin{enumerate}
   \item これがカカ型
   \item これがカキ型
   \item これがカク型
   \item これがカケ型
   \item これがカコ型
   \item これがカサ型
   \item これがカシ型
   \item これがカス型
   \item これがカセ型
   \item これがカソ型
\end{enumerate}
テンプレートに従った資料は, テンプレート内の任意の音素を発音する際の前後条件が全ての資料で等しくなるため, 調音運動を確認する際に適していることが考えられる. また, こちらは男女1名ずつ, 計2名の画像を用いて検討をおこなう.

\subsection{構成}
本稿では,  3章に正規化処理に関する目的, 手法, 結果, 考察が纏められている. これは, 1つ目のリサーチクエスチョンである "音素ごとの調音運動を比較するため, 背骨の傾きと顔の大きさの個人差を除去するためには, どの手法が適切か." に関する内容である.
4章では, 強調表示処理に関する目的, 手法, 結果, 考察が纏められている. これは, 2つ目のリサーチクエスチョンである "学習者が調音運動資料の何処に注目すれば良いのかを確認できるようにするためには,どの手法が適切か" に関する内容である.
5章では, 本研究のまとめと今後の課題に関して述べる.