本章では, 本研究のまとめと今後の課題を述べる.

\subsection{おわりに}
本研究では, 調音運動提示を用いた発話支援訓練で用いる調音運動資料の生成を目的とした. この目的を達成するために, 以下の2つのリサーチクエスチョンを設定した.

\begin{enumerate}
   \item 音素ごとの調音運動を比較するため, 背骨の傾きと顔の大きさの個人差を除去するためには, どの手法が適切か.
   \item 学習者が調音運動資料の何処に注目すれば良いのかを確認できるようにするためには, どの手法が適切か.
\end{enumerate}

まず, 音素ごとの運動を比較できるようにするために正規化処理を, 主成分分析を用いておこなった. その結果, 口蓋形状も含めた正規化を可能としたことにより, 口の開き具合や舌の位置の遷移などが, 異なる発話者のデータでも同様に比較することが可能になった. この成果により, 調音運動提示を用いた発話訓練場面において, 同一音素における複数の調音運動パターンを提示が可能となったと考えられる.

次に, 学習者が注目すべき運動を判別できるようにするために強調表示処理を, オプティカルフロー法を用いておこなった. 結果, 発話音素の違いによって異なる舌の動かし方をしている点が確認できる画像が生成された. したがって, rtMRI資料から調音運動の検出の自動化が可能になったと言える.

以上より本研究では, 発話者ごとに, 音ごとの舌の調音運動の違いを rtMRI 資料から比較できる成果が得られた. その結果を図\ref{final:result}に示す. この成果により, 発話訓練場面において, 調音に詳しくない学習者でも, 同一音素における調音運動パターンを確認できるようになった可能性が示唆される.

\begin{figure}[htbp]
  \begin{minipage}[b]{0.45\linewidth}
    \centering
    \includegraphics[keepaspectratio, width=0.8\linewidth]{opt_result/original_ka_16.png}
    \subcaption{元のrtMRI資料}
    \label{result:original}
  \end{minipage}
  \begin{minipage}[b]{0.45\linewidth}
    \centering
    \includegraphics[keepaspectratio, width=0.8\linewidth]{opt_result/ka/5_step_img_001_002.png}
    \subcaption{正規化処理と強調表示処理済みのrtMRI資料}
    \label{result:final}
  \end{minipage}
  \caption[処理前後の比較]{処理前後の比較. (a)は元のrtMRI資料, (b)は(a)の画像に正規化処理と強調表示処理を施したもの. (b)では何処が動いているのかが確認しやすくなっている. }
  \label{final:result}
\end{figure}

\subsection{今後の展望}
最終的には, 図\ref{siasyu_mokuhyou}で示したような実用場面で使える発話訓練システムの開発に取り組むことを考えている.
本研究の成果は, 図\ref{siasyu_mokuhyou}の左上端にあたる, 訓練のために提示する画像の生成をおこなった段階である.
そこで今後はユーザの声を分析して的確なフィードバックを返せるようにするために, 図\ref{siasyu_mokuhyou}の右上にあたる音響情報から調音運動を推定する手法の検討に取り組む予定である.